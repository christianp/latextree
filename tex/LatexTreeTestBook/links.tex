% !TEX root = main.tex

%--------------------
\chapter{Links}\label{ch:links}

Here is a labelled theorem containing a labelled equation.
\begin{theorem}[Pythagoras' Theorem]\label{thm:pythagoras}
\begin{equation}\label{eq:pythagoras}
a^2 + b^2 = c^2.
\end{equation}
\end{theorem}

%\begin{lemma}
%Statement of lemma.
%\end{lemma}
%\begin{corollary}
%Statement of corollary.
%\end{corollary}

%Labels are attached to the parent node. Any second label will overwrite the first.
\begin{itemize}
\item Chapter~\ref{ch:levels} is the levels chapter.
\item Chapter~\ref{ch:links} involves a reference to the current chapter.
\item Theorem~\ref{thm:pythagoras} involves a reference to the previous theorem.
\item Here is a ref -\ref{ch:levels}- to the levels chapter.
\item Here is a ref -\ref{ch:links}- to the current chapter.
\item Here is a ref -\ref{thm:pythagoras}- to the above theorem.
\item Here is an eqref -\eqref{eq:pythagoras}- to the equation in the above theorem.
\item Here is a citation -\cite{grimmett01}- to a bibtex entry.
\item Here is a url: -\url{http://www.bbc.co.uk/}-.
\item Here is some -\href{http://www.bbc.co.uk/}{hyperlinked text}-.
\item Here is a -\hyperref[ch:intro]{named cross-reference}- to the introduction.
\item here is an autoref -\autoref{ch:levels}- to the levels chapter.
\item Here is a nameref -\nameref{ch:levels}- to the levels chapter.
\item Here is a pageref -\pageref{ch:levels}- to the levels chapter.
\end{itemize}

\subsection*{Document-level labels}
For a viewable links to the document-level label we need to use the \texttt{hyperref} command as follows: -\hyperref[book:cameltest]{hello}-. There is no number or title associated with the document-level so other xref commands don't know what to display. Document-level labels might be useful to organise documents within and across modules.

\endinput

