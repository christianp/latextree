%------------------------------------------------
% main.tex - Test file for LatexTree package
%------------------------------------------------
\documentclass[12pt]{exam}

% uncomment to print solutions
\printanswers

% document info
\title{Test Exam for the LatexTree Package}
\author{D Evans}
\date{Summer 2017}

% ams packages
\usepackage{amsmath,amsfonts,amssymb}

% layout
\usepackage{geometry}
\geometry{paperwidth=210mm, paperheight=297mm	}
\geometry{left=25mm,right=40mm,top=30mm,bottom=30mm}
\setlength{\parindent}{0ex}
\setlength{\parskip}{3ex}

% new mathmode commands
\newcommand{\N}{\mathbb{N}}
\newcommand{\Z}{\mathbb{Z}}
\newcommand{\R}{\mathbb{R}}
\newcommand{\C}{\mathbb{C}}
\newcommand{\prob}{\mathbb{P}}
\newcommand{\expe}{\mathbb{E}}
\newcommand{\supp}{\text{supp}}
\newcommand{\var}{\text{Var}}
\newcommand{\cov}{\text{Cov}}
\DeclareMathOperator*{\argmin}{\arg\!\min}
\DeclareMathOperator*{\argmax}{\arg\!\max}

% TeX aliases (these are expanded by latextree prior to parsing).
\def\it{\item}
\def\bit{\begin{itemize}}
\def\eit{\end{itemize}} 
\def\ben{\begin{enumerate}}
\def\een{\end{enumerate}}

% exam.cls options
\bracketedpoints
\pointsinrightmargin
\pointsdroppedatright
\addpoints
	
% exam.cls tweaks
\renewcommand{\questionshook}{\addtolength{\itemsep}{3ex}}
\renewcommand{\labelenumi}{(\alph{enumi})}
\renewcommand{\subpartlabel}{(\thesubpart)}
\cfoot{\thepage}

% tweak choices and checkboxes environments
\CorrectChoiceEmphasis{\bfseries}
\newcommand{\bigsquare}{\raisebox{0.5ex}{\fbox{\phantom{\rule{0.5ex}{0.5ex}}}}}
\checkboxchar{$\bigsquare$}
\checkedchar{$\text{\rlap{\,\!$\checkmark$}}\bigsquare$}

% exam.cls aliases
\newcommand{\correct}{\correctchoice}
\newcommand{\incorrect}{\choice}

% macro/environment to record responses to choices
\newcommand{\responsetitle}{\noindent\textbf{Response:}\enspace}
\newcommand{\resp}[1]{\ifprintanswers\qquad\fbox{\responsetitle\normalfont #1}\fi}
\usepackage{environ}
\usepackage{mdframed}
\NewEnviron{response}{%
	\ifprintanswers
		\begin{mdframed}[roundcorner=5pt]
		\responsetitle
        \BODY%
		\end{mdframed}
    \fi
}{}
% These are only useful for interactive environments - let's kill them!
\renewcommand{\resp}[1]{}
\RenewEnviron{response}{}{}

% Define answers environment
% This mimics the built-in "solution" environment
% By default, LatexTree includes solutions (in expandable divs) but excludes answers.
\newcommand{\answertitle}{\noindent\textbf{Answer:}\enspace}
\NewEnviron{answer}{%
	\ifprintanswers
		\begin{mdframed}[roundcorner=5pt]
		\answertitle
        \BODY%
		\end{mdframed}
    \fi
}{}

	
% watermark solutions version
\AtBeginDocument{
\ifprintanswers
	\usepackage{draftwatermark}
	\SetWatermarkLightness{0.9}
	\SetWatermarkText{SOLUTIONS}
	\SetWatermarkAngle{60}
	\SetWatermarkScale{5}
	\rfoot{}{}
\fi
}

% define exercise and quiz environments
% amsthm does not produce a newline if the first line is the start of a list.
% We  must either put a space (\ ) after \begin{theorem}, or use ntheorem instead.
%\usepackage{amsthm}
%\newtheoremstyle{cumaths}{}{}{\upshape}{}{\bfseries}{~}{\newline}{}
%\theoremstyle{cumaths}
\usepackage[hyperref,thmmarks,amsmath]{ntheorem}
\theoremstyle{break}
\setlength\theorempreskipamount{3ex}
\setlength\theorempostskipamount{2ex}
\theorembodyfont{\upshape}
\qedsymbol{\rule{1ex}{1ex}}
\newtheorem{exercise}{Exercise}
\newtheorem{quiz}[exercise]{Quiz}

%----------------------------------------
\begin{document}\label{exam:latextreetest}
\maketitle

We define two new theorem types: {\tt exercises} and {\tt quizzes}. 

Note that {\tt exam.cls} is sub-classed from the basic {\tt article} class, so it's not possible to have chapters. This is annoying but the package author claims to have good reasons for doing so (something to do with counters). 

%{\tt exam.cls} can apparently be hacked as follows
%\scriptsize
%\bit
%\it Change \verb+\LoadClass{article}+ to \verb+\LoadClass{book}+
%\it Change \verb+\edef\@queslabel{question@\arabic{question}}+ to 
%\it[]\verb+\edef\@queslabel{question@\arabic{chapter}@\arabic{section}@\arabic{subsection}@\arabic{question}}+
%\it Change \verb+\edef\@partlabel{part@\arabic{question}@\arabic{partno}}+ to 
%\it[] \verb+\edef\@partlabel{part@\arabic{chapter}@\arabic{section}@\arabic{subsection}@\arabic{question}@\arabic{partno}}+
%\eit
%\normalsize

%==============================
\section{Exercises}
%==============================
Here is an exercise. Solutions are placed in {\tt solution} environments which are built-in to {\tt exam.cls}. These are displayed by {\tt LatexTree} in expandable divs.

\begin{exercise}
\begin{questions}

% Q1. PROBABILITY SPACES 
\question 
Let $\Omega$ be the sample space of some random experiment and let $\prob$ be a probability measure on subsets of $\Omega$.

\begin{parts}

%--------------------
\part %
Define the term \emph{sample space}. 
\begin{solution}
The sample space of a random experiments is the set of all possible outcomes. When the experiment is performed, exactly one outcome will occur.
\end{solution}

%--------------------
\part
Define the term \emph{probability measure}.
\begin{solution}
A probability measure is a function on subsets of $\Omega$ such that $\prob(\emptyset)=0$, $\prob(\Omega)=1$ and for any sequence of pairwise disjoint events $A_1,A_2,\ldots$ 
\[
\prob\left(\bigcup_{n=1}^{\infty}A_n\right)=\sum_{n=1}^{\infty}\prob(A_n) \qquad\text{(countable additivity).}
\]
\end{solution}

%--------------------
\part
Let  $A_1\subseteq A_2\subseteq \ldots$ be an increasing sequence of events. Show that
\[
\prob\left(\bigcup_{n=1}^{\infty} A_n\right) = \lim_{n\to\infty}\prob(A_n).
\]
\begin{solution}
The event $A=\cup_{n=1}^{\infty} A_n$ can be written as a disjoint union,
\[
A  = A_1 \cup (A_2\setminus A_1) \cup (A_3\setminus A_2) \cup \ldots
\]
By additivity,
\[
\prob(A) = \prob(A_1) + \prob(A_2\setminus A_1) + \prob(A_3\setminus A_2) + \ldots
\]
and because $A_n\subseteq A_{n+1}$,
\[
\prob(A_{n+1}\setminus A_n)=\prob(A_{n+1})-\prob(A_n).
\]
Hence
\begin{align*}
\prob(A)		& = \prob(A_1) + \big[\prob(A_2) - \prob(A_1)\big] + \big[\prob(A_3) - \prob(A_2)\big] + \ldots \\
		& = \lim_{n\to\infty} \prob(A_n)
\end{align*}
\end{solution}

%--------------------
\part
A fair coin is tossed repeatedly. Show that a head eventually occurs with probability one.
\begin{solution}
Let $A_n$ be the event that a head occurs during the first $n$ tosses, and let $A$ be the event that a head eventually occurs. Then $A_1\subset A_2\subset A_3,\ldots$ is an increasing sequence with 
\[
A=\bigcup_{n=1}^{\infty} A_n.
\]
If we assume that the tosses are independent, 
\[
\prob(A_n) = 1 - \prob(\text{no heads in the first $n$ tosses}) = 1 - (1/2)^n.
\]
By the continuity property of probability measures,
\[
\prob(A) 
	= \prob\left(\bigcup_{n=1}^{\infty}A_n\right)
	= \lim_{n\to\infty} \prob(A_n)
	= \lim_{n\to\infty} \left(1 - \frac{1}{2^n}\right) = 1.
\]
\end{solution}
\end{parts}

% Q2. EXPECTATION
\question 
\begin{parts}

%--------------------
\part
If $X$ and $Y$ are simple random variables, show that 
\[
\expe(aX+bY)=a\expe(X)+b\expe(Y)\quad\text{ for all $a,b\in\R$.}
\]
\begin{solution}
See lecture notes.
\end{solution}

%--------------------
\part
Let $X$ be a random variable with the following cumulative distribution function (CDF), 
\[
F(x) = \begin{cases}
	1 - \left(\theta/x\right)^{\alpha}	& \text{ for $x\geq\theta$,} \\
	0						& \text{ otherwise,}
	\end{cases}
\]
where $\alpha>0$ and $\theta>0$ are the parameters of the distribution.
\begin{subparts}
%--------------------
\subpart
Show that $\expe(X)$ is infinite when $\alpha \leq 1$.
\begin{solution}
The PDF is $f(x) = \alpha\theta^{\alpha}/x^{\alpha+1}$ for $x>\theta$ (and zero otherwise), so
\[
\expe(X) = \int_{-\infty}^{\infty}xf(x)\,dx = \alpha\theta^{\alpha}\int_{\theta}^{\infty}\frac{1}{x^{\alpha}}\,dx.
\]
This integral converges if and only if $\alpha > 1$.
\end{solution}
%--------------------
\subpart
Find $\var(X)$ when $\alpha > 2$.
\begin{solution}
\begin{align*}
\expe(X) 
	& = \alpha\theta^{\alpha}\int_{\theta}^{\infty}\frac{1}{x^{\alpha}}\,dx 
	= \alpha\theta^{\alpha}\left[\frac{-1}{(\alpha-1)x^{\alpha-1}}\right]_{\theta}^{\infty} 
	= \frac{\alpha\theta}{\alpha-1}. \\
\expe(X^2) 
	& = \alpha\theta^{\alpha}\int_{\theta}^{\infty}\frac{1}{x^{\alpha-1}}\,dx 
	= \alpha\theta^{\alpha}\left[\frac{-1}{(\alpha-2)x^{\alpha-2}}\right]_{\theta}^{\infty}
	= \frac{\alpha\theta^2}{\alpha-2}. \\
\var(X)
	& = \expe(X^2) - \expe(X)^2 
	= \frac{\alpha\theta^2}{(\alpha-1)^2(\alpha-2)}.
\end{align*}
\end{solution}
\end{subparts}
\end{parts}

\end{questions}
\end{exercise}

Here is another exercise, this time with points for each question (not yet implemented in {\tt LatexTree}).
\begin{exercise}
\begin{questions}

%--------------------
\question[7]
Let $X\sim\text{Exponential}(\lambda)$ where $\lambda>0$ is a rate parameter. Find the CDF of the random variable $Y=e^X$.
\droppoints
\begin{solution}
\bit
\it $g(x)=e^x$ is one-to-one and increasing over $[0,\infty)$
\it The inverse transformation is $g^{-1}(y) = \log y$.
\it $\supp(f_X)=[0,\infty)$ so $\supp(f_Y)=[1,\infty)$.
\eit
Because $g$ is increasing, for $y\geq 1$ we have
\[
F_Y(y) = F_X\big[g^{-1}(y)\big] = 1 - e^{-\lambda\log y} = 1 - 1/y^{\lambda}
\]
and zero otherwise.
\end{solution}

%--------------------
\question[4]
Let $X$ be a continuous random variable and let $F(x)$ denote its cumulative distribution function (CDF). If the inverse function $F^{-1}$ exists for all $x\in\R$, show that the random variable $U=F(X)$ is uniformly distributed on $[0,1]$.
\droppoints
\begin{solution}
Since $F(x)=\prob(X\leq x)$ is a CDF, we have that $F(x)\in [0,1]$ for all $x\in\R$.
In particular, $\prob(U<0)=0$ and $\prob(U>1)=0$.
\par
For $u\in[0,1]$, because the inverse $F^{-1}$ exists, we have 
\begin{align*}
\prob(U\leq u) 
	& = \prob\big(F(X)\leq u\big) \\
	& = \prob\big(X\leq F^{-1}(u)\big) \\
	& = F\big(F^{-1}(u)\big) \\
	& = u
\end{align*}
which is the CDF of the uniform distribution on $[0,1]$.
\end{solution}

%--------------------
\question[4]
Given a pseudo-random number from the $\text{Uniform}[0,1]$ distribution, show how this can be converted into a pseudo-random number from the $\text{Rayleigh}(\sigma)$ distribution, whose CDF is given by
\[
F(x) = \begin{cases}
	1 - e^{-x^2/2\sigma^2}	&\text{for $x\geq 0$,} \\
	0						& \text{ otherwise,}
	\end{cases}
\]
where $\sigma>0$ is a parameter of the distribution.
\droppoints
\begin{solution}
Let $X\sim\text{Rayleigh}(\sigma)$. To invert the CDF let $u=1-e^{-x^2/2\sigma^2}$. Then $x^2 = -2\sigma^2\log(1-u)$, and because $X$ takes only non-negative values,
\[
F^{-1}(u) = -\sigma\sqrt{2}\log(1-u)
\]
Applying this to a pseudo-random number $u$ from the $\text{Uniform}[0,1]$ distribution yields a pseudo-random number from the $\text{Rayleigh}(\sigma)$ distribution.
\end{solution}

\end{questions}
\end{exercise}

%==============================
\newpage
\section{Quizzes}
%==============================

{\tt exam.cls} implements multiple choice questions via the \texttt{choices} environment and multiple answer questions via the \texttt{checkboxes} environment. Here we put solutions in an {\tt answers} environment, which are not displayed by {\tt LatexTree}.

\begin{quiz}\label{quiz:frivolous}
\begin{questions} 

% true or false (implemented via multiple choice)
\question True or false. Necessity is the mother of invention.
\begin{choices}
\correct True 		
\incorrect False		
\end{choices}
\begin{answer}
You may disagree.
\end{answer}

% multiple choice
\question In what year did Columbus first cross the Atlantic?
\begin{choices}
\incorrect 1490 \resp{Sorry, better luck next time.}
\incorrect 1491 \resp{Sorry, better luck next time.}
\correct   1492 \resp{Correct, well done.}
\incorrect 1493 \resp{Sorry, better luck next time.}
\end{choices}
\begin{answer}
In 1492, Columbus sailed the ocean blue.
\end{answer}

% multiple answer
\question 
Which of the following were members of the Beatles?  
\noindent
\begin{checkboxes}
\correct John 
\correct Paul
\correct George
\incorrect Bingo 
\end{checkboxes}
\begin{answer}
Bingo was not a member of the Beatles.
\end{answer}

% checkboxes 
\question
\label{qu:series}
Which of the following series are convergent?
\begin{checkboxes}
\incorrect $\sum_{n=1}^{\infty}\frac{1}{\sqrt{n}}$
\incorrect $\sum_{n=1}^{\infty}\frac{1}{n}$ 
	\resp{This is the \emph{harmonic} series, which is divergent.}
\correct $\sum_{n=1}^{\infty}\frac{1}{n^2}$
\correct $\sum_{n=1}^{\infty}\frac{1}{n^3}$
\end{checkboxes}
\begin{answer}
The second of these is the \emph{harmonic} series, which is divergent.
\end{answer}
\end{questions}
\end{quiz}


\end{document}
%----------------------------------------


